\documentclass[10pt,a4paper]{report}
\usepackage[utf8]{inputenc}
\usepackage[russian]{babel}
\usepackage[OT1]{fontenc}
\usepackage{amsmath}
\usepackage{amsfonts}
\usepackage{amssymb}
\usepackage{graphicx}
\usepackage[left=2cm,right=2cm,top=1cm,bottom=2cm]{geometry}

\begin{document}
%page 1
	\begin{titlepage}
		\begin{center}
			% Header
			Санкт-Петербургский государственный политехнический университет Петра Великого \\
			Кафедра компьютерных систем и программных технологий \\[7cm]
			
			% Title
			\textbf{Отчет по лабораторной работе 1}\\[0.1cm]
			\textbf{Дисциплина:} Телекоммуникационные технологии \\[0.1cm]
			\textbf{Тема:} Работы с системой Latex \\[3cm]
		\end{center}
		
		\vfill
		
		% Names
		\flushleft{Выполнил студент гр. 33501/1} \hfill\parbox{8cm}{
			\hspace*{3cm}\hspace*{-0.8cm}\rule{3cm}{0.8pt} Поляков К.О.}\\[0.6cm]
		
		\flushleft{Преподаватель} \hfill\parbox{8cm}{
			\hspace*{3cm}\hspace*{-0.8cm}\rule{3cm}{0.8pt} Богач Н.А.}\\[0.6cm]
		
		\hfill\parbox{9cm}{\hspace*{3cm}``\rule{0.7cm}{0.8pt}''\rule{3cm}{0.8pt}~2017 г.}
		\vfill
		
		
		\begin{center}
			% Bottom of the page
			Санкт-Петербург \par
			2017 г.
		\end{center}
	\end{titlepage}
	%number of page
	\setcounter{page}{2}
	
%Page 2
\pagebreak

\chapter{Введение}
\section{Цель работы}
Ознакомление с программой для верстки отчетов TeXShop.
\section{Задача работы}
Научиться создавать титульный лист, разделы, списки и писать несложные формулы в среде верстки отчетов TeXShop.
\chapter{Выполнение работы}
\section{Написание математических формул}

\begin{enumerate}

\item
\begin{displaymath}
	lim_{n \to \infty}
	\sum_{k=1}^n \frac{25}{2*k^3} = \frac{\pi}{2}
\end{displaymath}
	
\item 
\begin{displaymath}
	x^{3}+\sum_{x=1}^8 \frac{x}{2}= \sqrt{617}
\end{displaymath}

\item 
\begin{displaymath}
	x(t) = A * sin(w*t+f)
\end{displaymath}

\item
\begin{displaymath}
	y[n]= \sum_{k=-M_{1}}^{M_{2}}\frac{1}{M_{1}+M_{2}+1}x[n-k]
\end{displaymath}

\end{enumerate}

Формула интеграла:
\begin{displaymath}
	\iint_{\frac{sin(x)}{x}+y^{2} = 1} f(x, y) dx dy 
\end{displaymath}

Теорема Парсеваля:
\begin{displaymath}
	\sum_{-\infty}^{\infty}|x[n]|^{2}=\frac{1}{2\pi}\int_{-\pi}^{\pi}|X(e^{j\omega})|^{2}d\omega
\end{displaymath}


\section{Интерпретация результатов}
\LaTeX очень удобен для описания математических формул в отчетах. Имеется множество команд, которые заметно облегчают ввод дробных выражений, констант, переменных, производных, интегралов, дифференциалов и других формул. Таким образом, можно без особы трудностей вводить формулы различной сложности.

%Page 3
\pagebreak

\chapter{Вывод}
В данной лабораторной работе я познакомился  с системой создания и редактирования текстов TeXShop и расширением \LaTeX. Рассмотрел команды этой среды, необходимые для создания отчетов.
Данный метод создания отчета поволяет намного быстрее и лаконичнее создавать математические формулы, автоматизированные отчеты и редактировать файл без особых аппаратных затрат.
Система в виде скриптового языка оказалась гораздо привлекательнее в плане комфорта создания отчетов. Трудности могут возникнуть только при руссификации и совместимости версий, но и они просто решаются, благодаря большой армии поклонников \LaTeX
\end{document}